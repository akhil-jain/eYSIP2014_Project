\documentclass[20pt]{report}

\usepackage{geometry}
 \geometry{
 a4paper,
 total={210mm,297mm},
 left=20mm,
 right=20mm,
 top=20mm,
 bottom=20mm,
 }
\usepackage{graphicx}
\usepackage{hyperref}
\hypersetup{
    colorlinks=true,
    linkcolor=black,
    citecolor=black,
    filecolor=black,
    urlcolor=blue,
}

\begin{document}

\hspace{2.5in}
\textbf{Personal Details}
\medskip

\textbf{Name : }
Akhil Jain
\medskip


\textbf{Email Id : }
akhil.jain1216@gmail.com
\medskip


\textbf{Mobile Number : }
+91-9986991450
\medskip


\textbf{Title of Project : }
Kinect Based Module Development
\medskip


\textbf{Duration of the internship : }
26 May 2014 to 10 July 2014
\medskip


\textbf{Summary of your contribution to the project : }

 My contribution to the project \textbf{Kinect Based Module Development} involves the design of the 
 
 following 12 experiments :
\begin{itemize}
\item E1\_Depth\_Tracking\_on\_screen
\item E2\_Depth\_Tracking\_on\_firebird
\item E3\_Camera\_fundamentals
\item E4\_Skeleton\_Tracking\_Fundamentals
\item E5\_Skeleton\_Tracking\_firebird\_left\_right
\item E6\_Skeleton\_Tracking\_firebird\_front\_back
\item E7\_Skeleton\_Tracking\_angle\_between\_joints
\item E9\_Skeleton\_Tracking\_Complete
\item E10\_Voice\_Recognition\_firebird
\item E12\_Depth\_Tracking\_on\_screen\_linux
\item E13\_Depth\_Tracking\_on\_firebird\_linux
\item E14\_Tilt\_Demo\_linux
\end{itemize}
 This was complemented with documentation supported by chapters 1, 3 and a part of 4 in the kinect manual. The final stage was completed with video demonstrations for the experiments designed.
 
My contribution to the project \textbf{Programming Firebird V on Linux} involves :
\begin{itemize}
\item Installation of Avrdude and other necessary software.
\item Compilation of a .c file for Avr platform.
\item Generation of .hex file for Avr platform.
\item Loading the .hex file on Atmega2560 using Avrdude.
\end{itemize}
 This was supported with documentation of the same in chapter 1 in the manual corresponding to the project.
\medskip

\vspace{0.5in}

\hspace{2.5in}
\textbf{Project Status Report}
\medskip

\textbf{Objective of the work : }
I chose the kinect project to get an insight into the kinect sensor and realising that interfacing of kinect to the firebird was a potential combination with multiple possible applications in a wide domain.

\medskip

\textbf{Scope of the work : }

Kinect Based Module Development -
\begin{itemize}


\item Installation of Kinect driver and softwares, on Windows and Linux.
\item Creating Video tutorials and documentation for the same.
\item Interfacing the Kinect sensor with FB V - Windows and Linux.
\item Creating Video tutorials and documentation for the same.
\item Designing and implementing a few basic experiments using FB V and Kinect.
\item Creating Video tutorials and documentation for the same.
\end{itemize}

Programming Firebird V on linux - 
\begin{itemize}


\item Write and compile C program for AVR based FB V robot on Linux.
\item Compile to generate .hex file.
\item Create Video tutorial explaining the process.
\item Load .hex file on robot using bootloader/AVRdude programmer from linux.
\item Create Video tutorial explaining the process.
\item Repeat the above steps for ARM based robot.
\end{itemize}

\medskip

\textbf{Results and Discussion : }
The completion of the project (Kinect based module development) provides an interface between the kinect sensor and the Firebird V robot with experiments ranging from easy to advanced level to give a basic understanding of programming and obtaining the data from the sensor. The major obstacles faced in the project was the unavailability of working literature for the current versions of  the used softwares. Most of the literature and online support codes were for the older versions and many of the inbuilt functions no longer had any support in the latest version.
However, one has to continuously keep track of the latest versions of the softwares and their supported functions for the kinect sensor. Any break in progress would result in loss of continuity and people working in future may have to begin from scratch to understand the supported functions. Hence, continuous work on this project is necessary.

The completion of the project (Programming Firebird V on linux) provides a platform for future work on Firebird V on the linux platform along with the flexibility of understanding the underlying terminal codes for compiling a source code, generating an executable file and loading the same on Firebird V robot.
\medskip



\textbf{Bugs (Kinect Based Module Development) :}
\begin{itemize}
\item Skeleton tracking at times in not Continuous.
\item Tracking a seated skeleton requires a definite posture.
\end{itemize}

\textbf{Bugs (Programming FB V on linux) :}

\begin{itemize}
\item brltty bug in 64 bit machines.
\item Explicit installation of libusb and libudev Libraries.
\item Abscence of the 3 math libraries.
\end{itemize}

\medskip

\textbf{Future Work : }
The current versions of the software on linux (libfreenect) doesnot support skeleton tracking and the audio templates available are not reliable and unpredictable. Hence, it would be an interesting propect to design codes once the support is available in future versions of libfreenect.

\medskip

\begin{thebibliography}{99}
\bibitem{download_sdk} \url{http://www.microsoft.com/en-in/download/details.aspx?id=40278}
\bibitem{openkinect} \url{http://openkinect.org/wiki/Getting\_Started}
\bibitem{visual} \url{http://www.visualstudio.com/en-us/products/visual-studio-express-vs.aspx}
\bibitem{skeleton1} \url{http://learning.codasign.com/index.php?title=Skeleton\_Tracking\_with\_the\_Kinect}
\bibitem{video} \url{https://www.youtube.com/watch?v=AwIlr98YZgk#t=31\&hd=1}
\bibitem{serial} \url{http://members.ee.net/brey/Serial.pdf}
\bibitem{skeleton2} \url{http://www.dotnetfunda.com/articles/show/2069/how-to-track-skeleton-joins-using-kinect}
\bibitem{skeleton3} \url{http://channel9.msdn.com/coding4fun/kinect/Tracking-skeleton-joins-code-sample}
\bibitem{skeleton4} \url{http://social.msdn.microsoft.com/Forums/en-US/3f73ac86-9793-4586-9eb3-3cf2aa55fc77/c-write-kinect-joint-positions-xyz-to-txt-file?forum=kinectsdk}
\bibitem{audio1} \url{http://stackoverflow.com/questions/14204902/record-audio-with-kinect}
\bibitem{audio2} \url{http://channel9.msdn.com/coding4fun/kinect/Introduction-to-Kinect-Speech-Recognition}
\bibitem{installation} \url{http://www.ladyada.net/learn/avr/setup-unix.html}
\bibitem{loading} \url{http://kamilskowron.pl/en/avr/ubuntu-avr-and-c-programming-microcontrollers-on-linuxatmega8/}
\bibitem{binutils}\url{http://ftp.gnu.org/gnu/binutils/}
\bibitem{avr-libc}\url{http://savannah.nongnu.org/projects/avr-libc/}
\bibitem{avrdude}\url{http://download.savannah.gnu.org/releases/avrdude/}
\end{thebibliography}

\end{document}